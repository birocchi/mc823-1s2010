% Exemplo de relatório técnico do IC
% Criado por P.J.de Rezende antes do Alvorecer da História.
% Modificado em 97-06-15 e 01-02-26 por J.Stolfi.
% Last edited on 2003-06-07 21:12:18 by stolfi

% modificado em 1o. de outubro de 2008

\documentclass[11pt,twoside]{article}
\usepackage{techrep-ic}
\usepackage[pdftex]{graphicx}
\usepackage{enumerate}

%%% SE USAR INGLÊS, TROQUE AS ATIVAÇÕES DOS DOIS COMANDOS A SEGUIR:
\usepackage[brazil]{babel}
%% \usepackage[english]{babel}

%%% SE USAR CODIFICAÇÃO LATIN1, TROQUE AS ATIVAÇÕES DOS DOIS COMANDOS A
%%% SEGUIR:
%% \usepackage[latin1]{inputenc}
\usepackage[utf8]{inputenc}

\begin{document}

%%% PÁGINA DE CAPA %%%%%%%%%%%%%%%%%%%%%%%%%%%%%%%%%%%%%%%%%%%%%%%
%
% Número do relatório
\TRNumber{45}

% DATA DE PUBLICAÇÃO (PARA A CAPA)
%
\TRYear{10} % Dois dígitos apenas
\TRMonth{03} % Numérico, 01-12

% LISTA DE AUTORES PARA CAPA (sem afiliações).
\TRAuthor{Birocchi, Anderson - RA: 072787 \and Braga, Felipe - RA:070803}

% TÍTULO PARA A CAPA (use \\ para forçar quebras de linha).
\TRTitle{MC823 - Laboratório de Redes\\Projeto 1: Servidor TCP Multi-usuário para Consulta a Banco de Dados de um Cinema}

\TRMakeCover

%%%%%%%%%%%%%%%%%%%%%%%%%%%%%%%%%%%%%%%%%%%%%%%%%%%%%%%%%%%%%%%%%%%%%%
% O que segue é apenas uma sugestão - sinta-se à vontade para
% usar seu formato predileto, desde que as margens tenham pelo
% menos 25mm nos quatro lados, e o tamanho do fonte seja pelo menos
% 11pt. Certifique-se também de que o título e lista de autores
% estão reproduzidos na íntegra na página 1, a primeira depois da
% página de capa.
%%%%%%%%%%%%%%%%%%%%%%%%%%%%%%%%%%%%%%%%%%%%%%%%%%%%%%%%%%%%%%%%%%%%%%

%%%%%%%%%%%%%%%%%%%%%%%%%%%%%%%%%%%%%%%%%%%%%%%%%%%%%%%%%%%%%%%%%%%%%%
% Nomes de autores ABREVIADOS e titulo ABREVIADO,
% para cabeçalhos em cada página.
%
\markboth{Birocchi, Braga}{MC823 - Projeto 1, Aplicação TCP}
\pagestyle{myheadings}

%%%%%%%%%%%%%%%%%%%%%%%%%%%%%%%%%%%%%%%%%%%%%%%%%%%%%%%%%%%%%%%%%%%%%%
% TÍTULO e NOMES DOS AUTORES, completos, para a página 1.
% Use "\\" para quebrar linhas, "\and" para separar autores.
%
\title{MC823 - Projeto 1, Servidor TCP Multi-usuário para Consulta a Banco de Dados de um Cinema}

\author{Anderson Birocchi, Felipe Braga}

\date{}

\newpage
\tableofcontents
\newpage

\maketitle

%%%%%%%%%%%%%%%%%%%%%%%%%%%%%%%%%%%%%%%%%%%%%%%%%%%%%%%%%%%%%%%%%%%%%%

\newenvironment{codelisting}
{\begin{list}{}{
\setlength{\leftmargin}{1em}
}
\item\scriptsize\bfseries}{\end{list}}


\begin{abstract}
Este projeto consiste da implementação de um sistema, que implementa comunicação em rede, baseado no paradigma cliente-servidor, motivado pela provisão de acesso a uma base de dados de filmes. O protocolo de transporte utilizado foi o TCP - \textit{Transmission Control Protocol}, e foram feitas análises quanto ao desempenho da aplicação considerando os atrasos de comunicação.
\end{abstract}

\section{Introdução}
Nos dia de hoje é comum se ter qualquer tipo de informação de forma rápida e prática através da internet, como mapas, receitas, notícias, preços de mercado, cotação das moedas do mundo todo, etc. De tudo que se pode fazer pela internet, a área que mais cresce é a de prestação de serviços, onde já se pode fazer compras, pagar contas, fazer transações bancárias dentre várias outras coisas, e um desses tipos de serviço será implementado pelo nosso projeto.\\
Nosso projeto irá implementar um servidor TCP multi-usuário para consultas em bancos de dados de um cinema, onde são armazenadas várias informações sobre os filmes em cartaz, como o título, sinopse, horário das sessões e as salas.\\
O intuito é de centralizar as informações dos filmes em um único local, evitando que cada cliente tenha que ficar se sincronizando toda vez que for fazer uma consulta; além de tornar o programa cliente leve e livre da carga de processamento necessário para as consultas, permitindo assim que aparelhos com poder de processamento menor como celulares ou palm-tops também possam executar o programa sem problemas.\\
Iremos criar os 2 aplicativos, o servidor e o cliente, e também criaremos o protocolo de comunicação entre os 2, e esse protocolo será baseado em sockets TCP programados em C.\\
A seção 2 irá especificar o que exatamente o programa deve fazer; a seção 3 irá comentar mais detalhadamente a implementação, as definições, suposições tomadas e ferramentas utilizadas; a seção 4 irá analisar o desempenho da aplicação atraveś de várias medidas de tempo; a seção 5 irá mostrar quão confiável e consistente é a aplicação; na seção 6 encontra-se uma breve conclusão sobre o projeto e, por último, na seção 7 estará o código fonte necessário para compilar e executar o servidor e o cliente.


\section{Casos de Uso}
Primeiramente, devemos especificar o que a aplicação deve fazer, portanto segue abaixo uma listagem de 6 ações que serão implementadas. Nota: as ações com (*) precisam receber um identificador numérico do filme como entrada.\\
As ações sempre serão tomadas pelo cliente, sendo o servidor apenas quem irá processar o pedido. O resultado da ação apenas é vista pelo cliente que a iniciou, deixando o servidor totalmente à parte do que está acontecendo do lado dos seus clientes.
\subsection{Listar todas as informações de todos os filmes}
Mostrar o ID, título, sinopse, horário das sessões e as salas de todos os filmes cadastrados no banco de dados do servidor.
\subsection{Listar ID e título de todos os filmes}
Fazer uma busca rápida de todos os títulos e seus IDs, mais utilizado para auxiliar em futuras consultas, e utilização dos próximos casos de uso.
\subsection{Listar todas as informações de um filme (*)}
Buscar todas as informações sobre o filme com o dado ID.
\subsection{Mostrar a sinopse de um filme (*)}
Mostrar a sinopse completa do filme com o dado ID.
\subsection{Mostrar a avaliação de um filme (*)}
Mostrar a quantidade de votos o filme teve e qual foi a pontuação obtida.
\subsection{Avaliar um filme (*)}
Dar uma pontuação ao filme com o dado ID.



\section{Implementação}
O sistema é desenvolvido na linguagem C, utiliza as bibliotecas de sockets TCP para fazer a comunicação pela rede, gerencia as várias conexões com threads da biblioteca pthreads.h e faz controle de exclusão mútua através de semáforos da biblioteca semaphore.h.\\
Primeiramente vamos explicar como foi implementado o banco de dados e suas operações, e depois explicar como é feita a comunicação entre o servidor e o cliente, qual protocolo seguem, como são distribuídas as conexões entre as threads e quando e como são criadas.\\
Basicamente, o funcionamento do sistema é: o servidor fica em um laço infinito, aguardando requisições de conexão; o cliente estabelece essa conexão; inicia-se uma sequencia de requisições do cliente e respostas por parte do servidor; cliente encerra o uso.\\
Para organizar a discussão sobre a implementação, vamos tratar de tópicos separamente.

\subsection{Dados}
\subsubsection{Armazenamento (banco de dados)}
Ao invés de usar um sistema de gerenciamento de banco de dados (SGDB) optamos, pela praticidade, armazenar todos os nossos dados em um arquivo de texto puro chamado "filmes.dat" onde os campos que contém os atributos de um filme são separados pelo caractere '@' . Um registro da base de dados tem o seguinte formato:\\
\textit{tamanho do registro@ID@avaliações@média@título@sinopse@sala@horários@}\\
Com relação aos tipos dos dados, \textit{tamanho do registro} e \textit{ID} são numéricos, inteiros, de tamanho variável com limites de 1024 para o primeiro e $10^{20}$ - 1 para o segundo. Vale ressaltar que estes são os dois únicos campos de tamanho fixo da representação, com 4 dígitos para o primeiro e 6 (incluindo o ponto decimal) para o segundo.\\
\textit{Número de avaliações} e \textit{média} são numéricos, com o primeiro inteiro até 9999 e o segundo, ponto flutuante de 0 a 999, com duas casas decimais (000.00 a 999.99).\\
\textit{Título} (30), \textit{Sinopse} (900), \textit{Sala} (20) e \textit{Horários} (20) são cadeias de caracteres (\textit{strings}), com seus respectivos tamanhos citados.\\
Segue um exemplo de uma entrada no banco de dados:\\
\textit{279@1@0000@000.00@Rei Leão@Sinopse do filme@sala10@10h, 12h30, 15h@}\\
Nota: Para facilitar a inserção de novos filmes no sistema, foi feito um script (add\_filme.sh) que pode ser visto na seção 7.1.
\subsubsection{Estruturas de dados}
Existem duas formas básicas de como os dados de um filme são representados em memória: como uma cadeia crua de caracteres e como uma estrutura chamada \textbf{filme}.\\
A primeira opção, mais simples, é usada na extração dos dados do arquivo e na transmissão destes pela rede.\\
Já a segunda, agrega as informações de forma mais coerente, facilitando a implementação de funções que precisam acessar os dados de um filme para, por exemplo, imprimi-los na tela. A definição desta estrutura pode ser vista no arquivo data\_access.h (biblioteca de especificações sobre o acesso aos dados), localizado na seção 7.2.\\

\subsection{Comunicação Cliente-Servidor}
Todas as funcionalidades aqui descritas encontram-se especificadas no arquivo internet.h (7.4) e implementadas em internet.c (7.5).\\
A fim de facilitar a utilização das funções de envio e recebimento de informações, foram implementadas as funções \textit{socket\_push\_char()}, \textit{socket\_pop\_char()}, \textit{socket\_push\_buffer()} e \textit{socket\_pop\_buffer()}. As duas primeiras se encarregam de enviar para e retirar da conexão com o socket, um único caractere. Como trabalhamos com tamanhos variáveis de atributos, é bastante útil comparar cada caractere lido da stream antes de fazer a próxima leitura.\\
Já as outras duas, uma envia outra recebe um buffer, uma cadeia de caracteres da conexão. Mais útil e eficiente para transportar grandes lotes de informações de uma só vez.\\
Analisando o número de \textit{send()}s e \textit{recv()}s, certamente a comunicação caractere por caractere é muito despendiosa, pois cada mensagem de 1 byte é empacotada e enviada às camadas mais baixas da rede. Certamente essa não foi a melhor escolha possível, mas facilitou em grande parte o trabalho de verificação do fim de cada atributo.

\subsection{Concorrência}
A despeito da referência do Beej's, decidimos por implementar a concorrência através do uso de threads, e fomos muito felizes nessa decisão. A implementação está dentro do arquivo server.c, na seção 7.7.\\
Dentro do main() do servidor, existe um laço infinito, cuja responsabilidade é, ao chegar uma nova conexão, alocar uma thread disponível para tratá-la (através da função trata\_thread()). O controle das threads disponíveis é feito através de um vetor global available\_thrs[].\\
A partir de impressões de mensagens na saída padrão, é possível verificar facilmente o comportamento multi-usuário do servidor. Além de aceitar conexões simultâneas, pode-se forçar mais conexões do que o disponível, o que faz com que o servidor recuse a conexão. Se o cliente quiser tentar mais uma vez posteriormente, ele pode.

\subsection{Exclusão Mútua}
Decidimos implementar a funcionalidade de avaliação de filme no nosso projeto, o que implicou na necessidade de prover uma maneira segura de garantir que threads concorrentes não fariam escritas no arquivo de modo inconsistente.\\
O cenário de inconsistência é o seguinte: a thread T1 lê o número de avaliações N0 e média M0 do arquivo e processa novos valores N1 e M1 a partir da avaliação do cliente. Então, antes de T1 atualizar os dados no arquivo, a thread T2 lê N0 e M0 arquivo e calcula seus valores N2 e M2. Aconteça o que acontecer, os dados estarão inconsistentes, pois as operações não estarão sendo cumulativas, como espera-se que sejam.\\
O meio de implementação da exclusão mútua entre as threads foi utilizando semáforos, conforme pode-se ver em server.c (7.7).\\
Como todas as threads funcionam com a mesma função, e foi possível fazer a separação clara entre cada caso de uso, foi simples estabelecer a região crítica (que não pode nunca ser executada por mais de uma thread ao mesmo tempo).\\
Além disso, sendo o caso de escrita concorrente em arquivo, o semáforo é iniciado com o valor 1, permitindo que apenas uma thread possa utilizar o recurso de cada vez.


\section{Análise de Tempo}
Para a análise de tempo, foram consideradas duas operações a serem analisadas: comunicação e consulta/atualização local.
\subsection{Tempo de Comunicação}
\subsubsection{Transmissão de caractere único}
\subsubsection{Transmissão de buffer}
\subsubsection{Transmissão de sequencia, caractere por caractere}

\subsection{Tempo de Operações Locais do Servidor}
\subsubsection{Tempo de Seleção de Registro no Arquivo}
\subsubsection{Tempo de Atualização de Média (Avaliação)}
\subsubsection{Tempo de Operação para server\_lista\_todos\_completo()}



\section{Confiabilidade e Consistência}
Dois pontos a considerar: robustez quanto à comunicação e consistência dos dados.\\
Tentamos fazer todas as verificações de erros possíveis nas funções das bibliotecas de comunicação com o socket, tanto no lado cliente como no servidor. Ainda assim, não conseguimos implementar um importante caso de exceção: notificar e encerrar o cliente quando o servidor "cai".\\
Com relação à consistência dos dados, esta é garantida pela exclusão mútua nas operações de atualização de avaliação.

\section{Conclusão}


\section{Código Fonte}

%%%%%%%%%%%%%%%%%%%%%%%%%%%%%%%%%%%%%%%%%%
%%%%%%% add_filme.sh
\subsection{add\_filme.sh}  % 1
\begin{verbatim}
#!/bin/bash

echo "Cadastro de novo filme!"
echo -n "Id: "; read id
echo -n "Titulo: "; read titulo
echo -n "Sinopse: "; read sinopse
echo -n "Sala: "; read sala
echo -n "Horários: "; read horarios

tam_substr=$(echo -n $id\@0000\@000\.00\@$titulo\@$sinopse\@$sala\@
$horarios\@ | wc -c)

# tam_reg deve ser tam_substr + 3~4 (digitos+@)
tam_reg=$(echo -n $tam_substr\@$id\@0000\@000\.00\@$titulo\@$sinopse\@
$sala\@$horarios\@ | wc -c)

# expressão regular ^10*$ casa com qq seq começando com 1 seguida de zeros
if [ $tam_reg = $(echo $tam_reg | grep ^10*$) ]
then
  echo 'Entrou no caso da exceção da virada 10/100/1000'
  tam_reg=$((tam_reg + 1))
else
  echo 'Não entrou na exceção.'
fi

# finalmente, manda os valores para o arquivo
echo -n $tam_reg\@$id\@0000\@000\.00\@$titulo\@$sinopse\@$sala\@
$horarios\@ >> filmes.dat

echo "Registro adicionado com sucesso!"
\end{verbatim}


%%%%%%%%%%%%%%%%%%%%%%%%%%%%%%%%%%%%%%%%%%
%%%%%%% data_access.h
\subsection{data\_access.h} % 2
\begin{verbatim}
#define TAM_MAX_REG 1024 /* 1kB */
#define TAM_MAX_TIT 30
#define TAM_MAX_SIN 900
#define TAM_MAX_SALA 20
#define TAM_MAX_HOR 20

#define TAM_MAX_ATR 900
/* tamanho max dos digitos do tamanho do registro e do id */
#define TAM_REG_ID 20 
/* abc.de (valor de 0 a 100, com 2 dígitos decimais) */
#define TAM_MEDIA 6 
/* cada filme pode ter até 9999 avaliações */
#define TAM_N_AVALIACOES 4 

typedef struct {
  int id;
  char *titulo;
  char *sinopse;
  char *sala;
  char *horarios;
  
  int n_aval; /* número de avaliações */
  float media;

  /* NULL caso seja um único filme ou o último da lista */
  struct filme *prox_filme; 
} filme;

/* 'da' refere-se a data access 
  (para facilitar depois pra chamar as funções) */

/* Função responsável por fazer um parse 
  da string lida do arquivo para um filme */
int da_str_to_filme(filme *f_ret, int *tam_reg, char *f_str);

/* Libera as strings alocadas dinamicamente */
void da_free_strs(filme *f);



/******* Funções a serem chamadas externamente (API) ********/

/* Imprime as informações já formatadas de um filme */
void da_print_full_info(filme *f);

/* Imprime as informações já formatadas de um filme */
void da_print_partial_info(filme *f);

/* Função que retorna um filme a partir de um id */
int da_get_filme_by_id(char *f_str, int id, int *tamanho);

/* Libera toda a memória alocada para o(s) filme(s) (strings e struct) 
 - retorna o numero de filmes desalocados */
int da_free_all(filme *f);

/* Seta uma lista com todos os filmes no arquivo */
int da_get_todos_filmes(filme **filmes);

/* Retorna o número de filmes no arquivo */
int da_get_n_filmes();

/* Retorna uma matriz com os registros todos em formato string pura */
void da_get_raw_strings (char **registros,
   int *tam_registros, int n_registros);

/* Avalia a nota de avaliação do filme e atualiza 
  seu número de atualizações */
int da_avalia_filme(int id, float nota);
\end{verbatim}


%%%%%%%%%%%%%%%%%%%%%%%%%%%%%%%%%%%%%%%%%%
%%%%%%% data_access.c
\subsection{data\_access.c} % 3
\begin{verbatim}
#include <stdio.h>
#include <stdlib.h>
#include "data_access.h"

int da_str_to_filme (filme *f_ret, int *tam_reg, char *f_str) {
  /* Entrada: string crua lida do arquivo, a partir do início de um registro */
  /* Saída: - setup da struct filme passada por referência 
     - tamanho do registro no arquivo
     - valor numérico para erros */
	

  /* Formato do registro no arquivo */
  /* Como há atributos de texto, é difícil manter os registros com 
     tamanho constante no arq */
  /* Assim, uma ideia é usar separadores entre os atributos, assim
     como manter o tamanho do registro */
  /* 
     int tam_total_do_reg (incluindo todos os separadores do registro)
     int id, int avaliacoes, float media [TAMANHO FIXO NO ARQUIVO!], 
     string titulo, string sinopse, 
     string sala, string horarios.
     Separador: @
     Ex: 75@1@0@0@Rei Leão@Sinopse@Kinoplex - sala 10@12h40, 15h, 17h20@
     (contando tudo, incluindo os caracteres de tam_total_do_reg e os separadores)

     Esse esquema do número de caracteres num campo é pra 
     possibilitar a busca entre os registros.
  */

  char str[TAM_MAX_ATR]; /* string fixa bem grande pra comportar qq atributo */

  int i = 0; /* índice de acesso de f_str */
  int j = 0; /* indice para montagem da string  */

  /* tamanho do registro */
  while(f_str[i]!='@') {
    str[j] = f_str[i];
    i++; j++;
  }
  str[j] = '\0';
  *tam_reg = atoi(str);

  /* id */
  i++; j = 0;
  while(f_str[i]!='@') { str[j] = f_str[i]; i++; j++; }
  str[j] = '\0'; 
  f_ret->id = atoi(str);

  /* numero de avaliações */
  i++; j = 0;
  while(f_str[i]!='@') { str[j] = f_str[i]; i++; j++; }
  str[j] = '\0';
  f_ret->n_aval = atoi(str);
	
  /* média */
  i++; j = 0;
  while(f_str[i]!='@') { str[j] = f_str[i]; i++; j++; }
  str[j] = '\0';
  f_ret->media = atof(str);
	
  /* titulo */
  i++; j = 0;
  while(f_str[i]!='@') {
    str[j] = f_str[i];
    i++; j++;
  }
  str[j] = '\0';
  /* aloca a string dinamicamente */
  f_ret->titulo = (char *)malloc((j+1)*sizeof(char));
  sprintf(f_ret->titulo, "%s", str);

  /* Blocos de código similares ao de cima, só que comprimidos  */
  /* sinopse */
  i++; j = 0;
  while(f_str[i]!='@') { str[j] = f_str[i]; i++; j++; } str[j] = '\0';
  f_ret->sinopse = (char *)malloc((j+1)*sizeof(char));
  sprintf(f_ret->sinopse, "%s", str);

  /* sala */
  i++; j = 0;
  while(f_str[i]!='@') { str[j] = f_str[i]; i++; j++; } str[j] = '\0';
  f_ret->sala = (char *) malloc((j+1)*sizeof(char));
  sprintf(f_ret->sala, "%s", str);

  /* horarios */
  i++; j = 0;
  while(f_str[i]!='@' && f_str[i]!='\0') { str[j] = f_str[i]; i++; j++; }
  str[j] = '\0';
  f_ret->horarios = (char *) malloc((j+1)*sizeof(char));
  sprintf(f_ret->horarios, "%s", str);

  /* next - lembra dele? :) */
  f_ret->prox_filme = NULL;

  return(0);
}


void da_print_full_info(filme *f) {
  printf("Id: %d\n", f->id);
  printf("Titulo: %s\n", f->titulo);
  printf("Sinopse: %s\n", f->sinopse);
  printf("Sala: %s\n", f->sala);
  printf("Horários: %s\n", f->horarios);
  printf("Média: %06.2f (%d avaliações)\n", f->media, f->n_aval);
  return;
}

void da_print_partial_info(filme *f) {
  printf("Id: %d\n", f->id);
  printf("Titulo: %s\n", f->titulo);
  return;
}


void da_free_strs(filme *f) {
  free(f->titulo);
  free(f->sinopse);
  free(f->sala);
  free(f->horarios);
  return;
}


int da_free_all(filme *f) {
  /* Esta função desaloca todos os filmes que estiverem na lista, e
     retorna o número desses filmes que foram desalocados.
     A lista precisa ser resolvida de trás pra frente, por isso 
     está sendo usada recursão. */

  int i;

  da_free_strs(f);

  /* Condição de parada (último filme) */
  if(f->prox_filme == NULL) {
    free(f);
    return(0); /* retorna, inicializando contador */
  }

  /* Chamada recursiva */
  i = da_free_all((filme *)f->prox_filme); /* cast pro -Wall n reclamar */

  /* Resolvida a recursão, libera a struct e retorna */
  free(f);
  return(i);

}
	

/* Função a ser chamada pelo SERVIDOR! */
int da_get_filme_by_id(char *f_str, int id, int *tamanho) {

  /* Função responsável por acessar o arquivo dos registros,
     buscar o filme com o id igual ao passado como argumento,
     e retornar o resultado da busca
     Saídas: 0 - filme encontrado (setado em f_str)
     1 - código de retorno que indica que nada foi encontrado
  */

  int tam_reg, id_reg;
  long int cursor = 0; /* indice de leitura do arquivo */
  FILE *arq;

  arq = fopen("filmes.dat", "r");

  while(fscanf(arq, "%d@%d@", &tam_reg, &id_reg) != EOF) {
    /* registro encontrado */
    if(id_reg == id) {
      /* caminha no arquivo até o inicio do registro */
      fseek(arq, cursor, SEEK_SET);
      fgets(f_str, tam_reg, arq);
      *tamanho = tam_reg;
      fclose(arq);
      return(0);
    } else {
      cursor += tam_reg;
      fseek(arq, cursor, SEEK_SET);
    }
  } /* só vai sair do while se não encontrar o filme */
  
  fclose(arq);
  
  return(1);

}


/* Função a ser chamada pelo SERVIDOR! */
int da_get_todos_filmes(filme **filmes_ret) {
  /* Esta função lê o arquivo, instancia uma struct filme para cada registro,
     concatena as structs, e retorna o resultado para o usuário.
     O valor de retorno é um inteiro que representa ou o número de
     registros lidos, ou -1 para erro.
  */

  FILE *arq;
  int tam_reg, i = 0;
  long int cursor = 0; /* indice de leitura do arquivo */
  char buffer[TAM_MAX_REG]; /* 1kB */
  filme *f; /* apontador principal que vai guardar cada um dos registros */

  arq = fopen("filmes.dat", "r");


  /* Enquanto houver registros no arquivo */
  while(fscanf(arq, "%d@", &tam_reg) != EOF) {
    /* Para cada registro, aloca a memória para a struct, seta o registro
     a partir da string lida, seta o cursor do arquivo p/ o próximo */
    i++;
    f = (filme *)malloc(sizeof(filme));
    if(i == 1) {
      /* Início da lista (primeiro registro) */
      *filmes_ret = f;
    }

    fseek(arq, cursor, SEEK_SET); /* pula p/ o inicio do registro */
    fgets(buffer, TAM_MAX_REG, arq); /* lê a string crua no buffer */
    da_str_to_filme(f, &tam_reg, buffer); /* abriga o filme na struct */

    cursor += tam_reg; /* ajuste para a leitura do próximo reg no arq */

    f = (filme *)f->prox_filme; /* atualiza o apontador */
  }

  f = NULL; /* último registro da lista */
  
  fclose(arq);
  
  return(i);
}


/* Retorna o número de filmes no arquivo */
int da_get_n_filmes() {

  FILE *arq;
  int tam_reg, i = 0;
  long int cursor = 0; /* indice de leitura do arquivo */

  arq = fopen("filmes.dat", "r");

  /* Enquanto houver registros no arquivo */
  while(fscanf(arq, "%d@", &tam_reg) != EOF) {
    i++;
    fseek(arq, cursor, SEEK_SET); /* pula p/ o inicio do registro */
    cursor += tam_reg; /* ajuste para a leitura do próximo reg no arq */
  }
  
  fclose(arq);
  
  return(i-1);
	
}

/* Retorna uma matriz com os registros todos em formato string pura */
void da_get_raw_strings (char **registros, int *tam_registros, int n_registros) {
  /* registros já é um vetor de apontadores pra strings, 
     cujo tamanho é o número de registros no arquivo. */
  FILE *arq;
  int tam_reg, i = 0;
  long int cursor = 0; /* indice de leitura do arquivo */

  arq = fopen("filmes.dat", "r");

  /* Enquanto houver registros no arquivo */
  while(fscanf(arq, "%d@", &tam_reg) != EOF) {
    /* Para cada registro, aloca e seta a string */
    registros[i] = (char *)malloc((tam_reg+1) * sizeof(char));
    fseek(arq, cursor, SEEK_SET); /* volta p/ o inicio do registro */
    fgets(registros[i], tam_reg, arq); /* lê a string crua */

    tam_registros[i] = tam_reg; /* seta também os tamanhos dos registros */

    cursor += tam_reg; /* ajuste para a leitura do próximo reg */
		
    i++;
  }
  
  fclose(arq);
  
  return;
}


/* Avalia a nota de avaliação do filme e atualiza seu número de atualizações */
int da_avalia_filme(int id, float nota) {
  /* Retorno
     1 - O filme com o id passado não existe
     0 - Atualização efetuada.
  */

  int tam_reg, id_reg, n_aval;
  float media;
  long int cursor = 0; /* indice de leitura do arquivo */
  FILE *arq;

  /* abre o arquivo com permissão para escrita */
  arq = fopen("filmes.dat", "r+");

  while(fscanf(arq, "%d@%d@%d@%f@", &tam_reg, &id_reg, 
    &n_aval, &media) != EOF) {
    /* registro encontrado */
    if(id_reg == id) {
      /* Cálculo da nova média para o filme */
      float nova_media;
      nova_media = (media*n_aval + nota)/(n_aval+1);
      printf("\n  thread diz: média antiga %06.2f; nova %06.2f\n",
         media, nova_media);
      
      /* caminha no arquivo até o início do número de avaliações */
      fseek(arq, -(2/*@s*/ + TAM_MEDIA + TAM_N_AVALIACOES), SEEK_CUR);

      /* Atualização dos valores: número de avaliações e média */
      fprintf(arq, "%04d@%06.2f@", n_aval+1, nova_media);

      fclose(arq);
      return(0);
    } else {
      cursor += tam_reg;
      fseek(arq, cursor, SEEK_SET);
    }
  } /* só vai sair do while se não encontrar o filme */
  
  fclose(arq);
  
  return(1);

}
\end{verbatim}


%%%%%%%%%%%%%%%%%%%%%%%%%%%%%%%%%%%%%%%%%%
%%%%%%% internet.h
\subsection{internet.h}     % 4
\begin{verbatim}
/*********************************************************************/
/*************************** Cliente *********************************/

/* Função auxiliar que retorna o socketfd da conexão com o servidor */
int client_get_connection(char **argv);

/* Função auxiliar de envio da opção para o servidor */
void client_send_option(int socketfd, char opt);

/* Le o número de filmes passado no próximo parâmetro da stream */
int client_get_n_filmes(int socket);

/* Copia a str de filme da stream */
void client_get_filme_str(int socket, char *f_str);


/*************************** Cliente *********************************/
/*********************************************************************/



/*********************************************************************/
/**************************** Server *********************************/

/* Recebe a opção da stream */
char server_recv_option(int connect_socketfd);

/**************************** Server *********************************/
/*********************************************************************/



/*********************************************************************/
/***************************** Geral *********************************/

/* Envia um caractere para a stream */
void socket_push_char(int socket, char c);

/* Retira um caractere da stream */
char socket_pop_char(int socket);

/* Envia um buffer para a stream */
void socket_push_buffer(int socket, int n, char *buffer);

/* Retira um buffer da stream */
void socket_pop_buffer(int socket, int n, char *buffer);


/***************************** Geral *********************************/
/*********************************************************************/
\end{verbatim}


%%%%%%%%%%%%%%%%%%%%%%%%%%%%%%%%%%%%%%%%%%
%%%%%%% internet.c
\subsection{internet.c}     % 5
\begin{verbatim}
#include "internet.h"
#include "defines.h"
#include "data_access.h"
#include <string.h>
#include <stdio.h>
#include <stdlib.h>
#include <netdb.h>

#include <sys/types.h>
#include <sys/socket.h>
#include <netinet/in.h>
#include <unistd.h>


/*********************************************************************/
/*************************** Cliente *********************************/

/* Função auxiliar que retorna o socketfd da conexão com o servidor */
int client_get_connection(char **argv) {

  int status, socketfd;
  struct addrinfo opcoes;
  struct addrinfo *servinfo;  // will point to the results

  memset(&opcoes, 0, sizeof(opcoes)); // zera a estrutura
  opcoes.ai_family = AF_INET;         // IPv4
  opcoes.ai_socktype = SOCK_STREAM;   // TCP stream sockets
  opcoes.ai_flags = AI_PASSIVE;       // fill in my IP for me

  status = getaddrinfo(argv[1], SERVER_PORT_STR, &opcoes, &servinfo);
  if (status != 0) {
    fprintf(stderr, "getaddrinfo error: %s\n", gai_strerror(status));
    exit(1);
  }

  /* cria o socket com os parâmetros setados */
  socketfd = socket(servinfo->ai_family, servinfo->ai_socktype,
    servinfo->ai_protocol);

  /* faz a conexão com o socket do servidor */
  status = connect(socketfd, servinfo->ai_addr, servinfo->ai_addrlen);

  /* Caso dê algum erro na conexão, pára o cliente */
  if (status == -1){
    fprintf(stderr, "Problema na conexão.\n");
    exit(1);
  }

  freeaddrinfo(servinfo); // libera a estrutura de informações do servidor
  return(socketfd);

}

/* Função auxiliar de envio da opção para o servidor */
void client_send_option(int socketfd, char opt) {
  socket_push_char(socketfd, opt);
  return;
}

/* Le o número de filmes passado no próximo parâmetro da stream */
int client_get_n_filmes(int socket) {

  char str[10], c;
  int i = 0;

	/* Limpa o buffer do canal de comunicação */
  c = socket_pop_char(socket);
	while(c=='\0') c = socket_pop_char(socket);

  /* Preenche uma str com os numeros até chegar o '@' */
  while (c != '@') {
    str[i] = c;
    i++;
    c = socket_pop_char(socket);
  }
  str[i] = '\0';
	
  return(atoi(str));
}

/* Copia a str de filme da stream */
void client_get_filme_str(int socket, char *f_str) {
	
  int i, tam_reg;
  char c, tam_reg_str[TAM_REG_ID];

  /* leitura do tamanho do registro */
  i = 0;
  c = socket_pop_char(socket);
  while (c != '@') {
    //f_str[i] = c;
    tam_reg_str[i] = c;
    i++;
    c = socket_pop_char(socket);
  }
  //f_str[i] = '@';
  tam_reg_str[i] = '\0';

  tam_reg = atoi(tam_reg_str);

  /* Sei o tamanho do registro e sei até onde já li; vou ler o resto */
  tam_reg = tam_reg - i -1; /* numero de caracteres restantes */
  int n = 0;
  char buffer[TAM_MAX_ATR];
  while (n < (tam_reg-1)) {
    n += recv(socket, &buffer[n], (tam_reg - n), 0);
  }
  buffer[n] = '\0';
	
  /* por fim, concatena as strings já lidas, copiando para a str de retorno */
  sprintf(f_str, "%d@%s", (tam_reg+i+1), buffer);
	
  return;
}

/*************************** Cliente *********************************/
/*********************************************************************/


/*********************************************************************/
/**************************** Server *********************************/

/* Recebe a opção da stream */
char server_recv_option(int connect_socketfd) {
  return(socket_pop_char(connect_socketfd));
}


/**************************** Server *********************************/
/*********************************************************************/


/*********************************************************************/
/***************************** Geral *********************************/

/* Envia um caractere para a stream */
void socket_push_char(int socket, char c) {
  int n = 0;
	
  while (n != sizeof(char)) {
    n = send(socket, &c, sizeof(char), 0);
  }

  return;
}

/* Retira um caractere da stream */
char socket_pop_char(int socket) {
  char c;
  int n = 0;

  while (n != sizeof(char)) {
    n = recv(socket, &c, sizeof(char), 0);
  }

  return(c);
}


/* Envia um buffer para a stream */
void socket_push_buffer(int socket, int n, char *buffer) {
  /* 
     Entradas:
     n - número de caracteres a serem escritos;
     buffer - buffer de onde se lê.
  */

  int i = 0;
  
  while(i < (n-1)) {
    i += send(socket, &buffer[i], (n - i), 0);
  }

  return;
}

/* Retira um buffer da stream */
void socket_pop_buffer(int socket, int n, char *buffer) {
  /* 
     Entradas:
     n - número de caracteres a serem lidos;
     buffer - buffer de leitura.
  */

  int i = 0;

  /* Vai acumulando o valor dos bytes já lidos, e enquanto não
   chega ao fim, continua lendo... */
  while(i < (n-1)) {
    i += recv(socket, &buffer[i], (n - i), 0);
  }

  return;
}

/***************************** Geral *********************************/
/*********************************************************************/
\end{verbatim}


%%%%%%%%%%%%%%%%%%%%%%%%%%%%%%%%%%%%%%%%%%
%%%%%%% defines.h
\subsection{defines.h}      % 6 
\begin{verbatim}
#define TRUE 1
#define FALSE 0

#define LISTAR_TODOS_COMPLETO '1'
#define LISTAR_TODOS '2'
#define REG_COMPLETO '3'
#define REG_SINOPSE '4'
#define REG_MEDIA '5'
#define REG_AVALIAR '6'
#define SAIR '7'


// A Porta em que o servidor escuta e na qual os clientes irão se conectar
#define SERVER_PORT 50000
#define SERVER_PORT_STR "50000"

// Tamanho do Buffer de recepcao de mensagens
#define TAM_BUFFER 200

//Tamanho do Buffer de envio de Mensagens
#define TAM_MENSAGEM 200


// Estrutura para definir o atributo a ser passado para a thread
typedef struct {
  int connect_socket;
  int thr_index;
} thread_attr ;
\end{verbatim}


%%%%%%%%%%%%%%%%%%%%%%%%%%%%%%%%%%%%%%%%%%
%%%%%%% server.c
\subsection{server.c}       % 7
\begin{verbatim}
// Bibliotecas comuns
#include <string.h>
#include <stdio.h>
#include <stdlib.h>
#include <netdb.h>
#include "defines.h"
#include "internet.h"
#include "data_access.h"
#include <signal.h>

// Biblioteca para threads
#include <pthread.h>

// Exclusão Mútua
#include <semaphore.h>

// Bibliotecas para manipulacao de sockets
#include <sys/types.h>
#include <sys/socket.h>
#include <netinet/in.h>
#include <unistd.h>

// número máximo de possíveis conexões pendentes na fila
#define QTDE_CONEXOES 10 
/* valor maximo de threads simultaneas */
#define PTHREAD_THREADS_MAX 10  


/* Variável global que armazena o status para cada thread
   TRUE: significa que ela está disponível;
   FALSE: significa que ela não está disponível.
*/
int available_thrs[QTDE_CONEXOES];


/* Semáforo global usado para garantir exclusão mútua entre as threads 
   no uso do arquivo (escrita). */
sem_t semaphore;


/**************************************************************/
/******[inicio] Funções que implementam os casos de uso  ******/

/* ## 1 ## */
void server_lista_todos_completo(int socket) {

  /*
    Esta função envia ao cliente uma sequencia de caracteres no formato:
    n_filmes@str_filme1@str_filme2@...@
  */

  /* Envio do numero de filmes */
  int n_filmes, n, i;
  char n_filmes_str[10]; /* max: 999999999@ */

  n_filmes = da_get_n_filmes();
  sprintf(n_filmes_str, "%d@", n_filmes);

  n = 0;
  while (n < strlen(n_filmes_str)) {
    n = send(socket, n_filmes_str, strlen(n_filmes_str), 0);
  }

  /* Agora, para cada filme, envia sua string crua. */
  char **registros;
  int *tam_reg;
  registros = (char **)malloc(n_filmes*sizeof(char *));
  tam_reg = (int *)malloc(n_filmes*sizeof(int));
  da_get_raw_strings(registros, tam_reg, n_filmes);

  for (i = 0; i < n_filmes; i++) {
    n = 0;
    socket_push_buffer(socket, tam_reg[i], registros[i]);
    free(registros[i]);
  }
  free(registros);
  free(tam_reg);

  return;
}

/* ## 2 ## */
void server_lista_todos(int socket) {
  /* O servidor faz exatamente a mesma coisa para a 
   listagem parcial. Todas informações são passadas
   para o cliente, mas este só imprime algumas delas. */
  server_lista_todos_completo(socket);
  return;
}

/* ## 3 ## */
void server_reg_completo(int socket) {
  
  /* servidor lê o ID que está sendo passado */
  char c, id_procurado[TAM_REG_ID]; /* 20 */
  int i = 0, tam_reg;

  /* leitura do ID pesquisado pelo cliente */
  c = socket_pop_char(socket);
  while (c!='@') {
    id_procurado[i] = c;
    c = socket_pop_char(socket);
    i++;
  }
  id_procurado[i] = '\0';

  int id;
  id = atoi(id_procurado);

  printf("  id requisitado: %d\n", id);

  /* Função que faz a busca.
     Retorna 1 se n encontrou nenhum filme.
     Caso contrário, aloca a memória e seta o filme. */
  
  char f_str[TAM_MAX_REG];

  /* se não encontrou nenhum filme, envia erro ao cliente */
  if (da_get_filme_by_id(f_str, id, &tam_reg) == 1) {
    socket_push_char(socket, '#');
    return;
  }

  /* se encontrou... */
  /* envia caractere de confirmação */
  socket_push_char(socket, '%');
  
  /* envia o filme */
  socket_push_buffer(socket, tam_reg, f_str);

  return;
}

/* ## 4 ## */
void server_reg_sinopse(int socket) {
  /* Neste caso, o servidor faz exatamente o mesmo  
   * que na listagem completa de um registro:
   * 1- Recebe o registro procurado;
   * 2- Faz a busca;
   * 3- Se não encontrar o filme, retorna o caractere '#'
   * 4- Se encontrar, retorna a string crua do filme
   */
  server_reg_completo(socket);
  return;
}

/* ## 5 ## */
void server_reg_media(int socket) {
  /* Neste caso, o servidor faz exatamente o mesmo  
   * que na listagem completa de um registro:
   * 1- Recebe o registro procurado;
   * 2- Faz a busca;
   * 3- Se não encontrar o filme, retorna o caractere '#'
   * 4- Se encontrar, retorna a string crua do filme
   */
  server_reg_completo(socket);
  return;
}

/* ## 6 ## */
void server_reg_avalia(int socket) {

  /* servidor lê o ID que está sendo passado */
  char c, id_avaliar[TAM_REG_ID] /*20*/, nota_s[7];
  int i = 0;

  /* leitura do ID requisitado pelo cliente p/ avaliação */
  do { 
    c = socket_pop_char(socket);
  } while (c=='\0'); /* limpa stream */
  while (c!='@') {
    id_avaliar[i] = c;
    c = socket_pop_char(socket);
    i++;
  }
  id_avaliar[i] = '\0';

  int id;
  id = atoi(id_avaliar);

  /* leitura da nota */
  i = 0;
  do { 
    c = socket_pop_char(socket);
  } while (c=='\0'); /* limpa stream */
  while (c!='@') {
    nota_s[i] = c;
    c = socket_pop_char(socket);
    i++;
  }
  nota_s[i] = '\0';

  float nota;
  nota = atof(nota_s);

  printf("  id p/ avaliar: %d\n", id);

  printf("  nota enviada: %06.2f", nota);

  /* 
   * Importante: Uso do semáforo para controle de concorrência
   * do recurso (no caso o arquivo), garantido exclusão mútua,
   * isto é, apenas uma thread poderá escrever nele por vez.
   */
  sem_wait(&semaphore); /* trava até o semáforo estar liberado */

  /**************************************/
  /* [Início] Região com Exclusão Mútua */
  int status;
  /* sleep(10); */ /* testes */
  
  /*
   * Função que realiza a avaliação:
   * Retorna 1 se o filme não existe.
   * Retorna 0 se ocorreu tudo bem.
   */
  status = da_avalia_filme(id, nota);
  /** [Fim] Região com Exclusão Mútua ***/
  /**************************************/

  sem_post(&semaphore); /* libera o semáforo */

  /* se não encontrou nenhum filme, envia erro ao cliente */
  if (status == 1) {
    socket_push_char(socket, '#');
  } else {
    /* se encontrou, envia caractere de confirmação */
    socket_push_char(socket, '%');
  }

  return;
}

/*******[fim] Funções que implementam os casos de uso  ********/
/**************************************************************/


/* Implementa o comportamento de cada thread */
void *trata_conexao (void *a) {
  
  int connect_socketfd;
  int t;

  char option;

  /* cast pra dizer que é um ap pra thread_attr */
  connect_socketfd = ((thread_attr *)a)->connect_socket;
  t = ((thread_attr *)a)->thr_index;

  /* recebe a opção enviada pelo cliente. */
  option = server_recv_option(connect_socketfd);
	
  printf("opção enviada pelo cliente: %c\n", option);

  /* Verificação do caso de saída e chamadas para cada caso específico */
  while(option != SAIR) {
      
    switch(option) {
			
    case LISTAR_TODOS_COMPLETO:
      server_lista_todos_completo(connect_socketfd);
      break;
    case LISTAR_TODOS:
      server_lista_todos(connect_socketfd);
      break;
    case REG_COMPLETO:
      server_reg_completo(connect_socketfd);
      break;
    case REG_SINOPSE:
      server_reg_sinopse(connect_socketfd);
      break;
    case REG_MEDIA:
      server_reg_media(connect_socketfd);
      break;
    case REG_AVALIAR:
      server_reg_avalia(connect_socketfd);
      break;
    }

    /* recebe a opção enviada pelo cliente. */
    option = server_recv_option(connect_socketfd);
    printf("opção enviada pelo cliente: %c\n", option);
  }
	
  printf("Thread %d diz: terminei! Fechando o socket %d...\n",
    t, connect_socketfd);
  close(connect_socketfd);

  /* thread seta o valor do vetor de disponíveis para TRUE novamente */
  available_thrs[t] = TRUE;

  pthread_exit(NULL);
}

//Trata o sinal de interrupcao mandar uma mensagem antes de encerrar
void trata_SIGINT(int sig) {
  printf("\nEncerrando o servidor...\n");
  /* Mesmo usando o socket de listen como variável global, e dando
     um close(list_socket) aqui, continua com o problema do bind
     depois de interromper o servidor. Isso acontece por causa da
     implementação do TCP no Kernel, que demora algum tempo pra
     liberar novamente a porta para bind. */

  /* Desaloca recursos para o semáforo */
  sem_destroy(&semaphore);

  exit(0);
}

int main() {

  // Sockets de escuta e de conexao
  int listen_socketfd, connect_socketfd; 

  signal(SIGINT,trata_SIGINT);

  int status;
  struct addrinfo opcoes;
  struct addrinfo *servinfo;  // Informações do meu endereço

  memset(&opcoes, 0, sizeof(opcoes)); // zera a estrutura
  opcoes.ai_family = AF_INET;         // IPv4
  opcoes.ai_socktype = SOCK_STREAM;   // TCP
  opcoes.ai_flags = AI_PASSIVE;       // fill in my IP for me

  status = getaddrinfo(NULL, SERVER_PORT_STR, &opcoes, &servinfo);
  if (status != 0) {
    fprintf(stderr, "getaddrinfo error: %s\n", gai_strerror(status));
    exit(1);
  }

  // cria o socket TCP de escuta
  listen_socketfd = socket(servinfo->ai_family,
    servinfo->ai_socktype, servinfo->ai_protocol);
  printf("Socket TCP de escuta criado!\n");
  
  //Atribui a porta utilizada ao socket de escuta
  status = bind(listen_socketfd, servinfo->ai_addr,
    servinfo->ai_addrlen); 
  if (status == -1) {
    fprintf(stderr, "error on binding the socket to a port\n");
    exit(1);
  }
  freeaddrinfo(servinfo); // libera a estrutura de informações do servidor

  /* Atribui o socket como ouvinte das conexões. */
  listen(listen_socketfd, QTDE_CONEXOES);

  /* Inicializa o semáforo. O número de recursos compartilhados 
     é 1 (apenas uma thread pode usar o arquivo para escrita de
     cada vez) - este é o terceiro argumento. O segundo argumento
     é uma flag com valor padrão 0. */
  sem_init(&semaphore, 0, 1);


  /* Threads Time! */
  //  pthread_t thread;
  pthread_t threads[QTDE_CONEXOES];
  int i, t;

  /* Inicializa o vetor de threds disponíveis */
  for (i = 0; i < QTDE_CONEXOES; i++) available_thrs[i] = TRUE;

  struct sockaddr_storage client_addr;
  socklen_t addr_size;

  while(TRUE) {
    /* Aceitação da conexão. */
    printf("Esperando alguma conexao...\n");
    connect_socketfd = accept(listen_socketfd,
      (struct sockaddr *)&client_addr, &addr_size);
    if (connect_socketfd == -1){
      printf("Problema na conexão.\n");
      continue; /* Desiste dessa conexão. */
    }
    printf("Conexao aceita!\n");
    
    /* Seleciona a primeira thread disponível */
    /* t recebe o indice da primeira thread disponível */
    t = -1;
    for (i = 0; i < QTDE_CONEXOES; i++) {
      if (available_thrs[i]==TRUE) {
	t = i; break;
      }
    }
    
    /* caso não haja nenhuma thread disponível, fecha a conexão e volta a ouvir */
    if (t == -1) { 
      printf("Não há thread disponível para aceitar a conexão.\n");
      close(connect_socketfd); continue;
    }

    /* Inicia a thread e passa o connect_socketfd pra ela. */
    available_thrs[t] = FALSE; /* agora essa thread não está disponível */
    printf("Main diz: vou passar para a thread o socket %d.\n", connect_socketfd);
    /* inicializa o atributo a passar para a thread */
    thread_attr a;
    a.connect_socket = connect_socketfd;
    a.thr_index = t;
    pthread_create(&threads[t], NULL, trata_conexao, (void *)&a);

  }

  return(0);
}
\end{verbatim}


%%%%%%%%%%%%%%%%%%%%%%%%%%%%%%%%%%%%%%%%%%
%%%%%%% client.c
\subsection{client.c}       % 8
\begin{verbatim}
//Bibliotecas comuns
#include <string.h>
#include <stdio.h>
#include <stdlib.h>
#include <netdb.h>   //Para usar o gethostbyname
#include "data_access.h"
#include "defines.h"
#include "internet.h"

//Bibliotecas para manipulacao de sockets
#include <sys/types.h>
#include <sys/socket.h>
#include <netinet/in.h>
#include <unistd.h>


/* Função auxiliar para tratamento de entrada */
char read_option() {
  /* considera os possiveis erros e só sai quando o usuario
     digitar um caractere válido */

  char c, aux;

  while(TRUE) {

    /* Mensagem com as opções... */
    system("clear");
    printf("Escolha uma entre as opções e tecle Enter:\n");
    printf("(opções com (*) requererão o id do filme)\n");
    printf(" [1] Listar todas as informações de todos os filmes.\n");
    printf(" [2] Listar id e título de todos os filmes.\n");
    printf(" [3] Listar todas as informações de um filme. (*)\n");
    printf(" [4] Mostrar a sinopse de um filme. (*)\n");
    printf(" [5] Mostrar a avaliação de um filme. (*)\n");
    printf(" [6] Avaliar um filme! (*)\n");
    printf(" [7] Sair\n  ");

    c = getchar(); aux = getchar();
    if((c==SAIR || c==LISTAR_TODOS_COMPLETO || c==LISTAR_TODOS ||
	c==REG_COMPLETO || c==REG_SINOPSE || c==REG_MEDIA ||
	c==REG_AVALIAR) && aux=='\n') {
      /* se a opcao lida é válida e o próximo caractere foi Enter, 
	 retorna o caractere válido digitado*/
      return(c);
    } else {
      /* caso contrário, 'come' todo o resto da linha e volta às msgs */
      if(aux != '\n') { while(getchar()!='\n'); }
    }
  } /* fim do while */

}



/**************************************************************/
/******[inicio] Funções que implementam os casos de uso  ******/

/* ## 1 ## */
void client_lista_todos_completo(int socketfd) {
  /* 
     Não é necessário enviar mais informações ao servidor, apenas
     aguardar um retorno.
     O formato desse retorno será:
     n_filmes@str_do_filme1@str_do_filme2@str_do_filme3@
  */

  int n_filmes, i;
  char filme_str[TAM_MAX_REG]; /* 1024 */
  filme *lista_filmes, *f, *last_f;
	
  /* Leitura do número de filmes retornado pelo servidor */
  n_filmes = client_get_n_filmes(socketfd);

  if (n_filmes == 0) {
    printf("Não há filmes no servidor!\n");
    printf("Tecle Enter para continuar...");
    getchar();
    return;
  }
  printf("Número de filmes encontrados: %d\n\n", n_filmes);

  for (i = 0; i < n_filmes; i++) {

    int tam_filme;
		
    /* Le a string do filme */
    client_get_filme_str(socketfd, filme_str);

    /* Aloca a estrutura para guardar o filme */
    f = (filme *)malloc(sizeof(filme));
    da_str_to_filme(f, &tam_filme, filme_str);

    /* Seta a lista dos filmes */
    if (i==0){ /* primeiro filme */
      lista_filmes = f;
    }
    else {
      last_f->prox_filme = f; /* concatena o filme à lista */
    }
    last_f = f; /* atualiza o apontador para o último filme */
  }
	
  /* para cada filme na lista, chama da_print_full_info(f) */
  for (f = lista_filmes; f != NULL; f = (filme *)f->prox_filme) {
    da_print_full_info(f);
    printf("\n-------------------------\n");
  }
	
  /* libera a memória dos filmes */
  da_free_all(lista_filmes);

  printf("Tecle Enter para continuar...");
  getchar();

  return;
}

/* ## 2 ## */
void client_lista_todos(int socketfd) {

  /* Cliente praticamente igual para a listagem completa. */
  int n_filmes, i;
  char filme_str[TAM_MAX_REG]; /* 1024 */
  filme *lista_filmes, *f, *last_f;
	
  n_filmes = client_get_n_filmes(socketfd);

  if (n_filmes == 0) { printf("Não há filmes no servidor!\n");
    printf("Tecle Enter para continuar..."); getchar(); return;
  }
  printf("Número de filmes encontrados: %d\n\n", n_filmes);

  for (i = 0; i < n_filmes; i++) {

    int tam_filme;
    client_get_filme_str(socketfd, filme_str);
    f = (filme *)malloc(sizeof(filme));
    da_str_to_filme(f, &tam_filme, filme_str);
    if (i==0) { lista_filmes = f;} else { last_f->prox_filme = f; }
    last_f = f; /* atualiza o apontador para o último filme */
  }
	
  /* para cada filme na lista, chama da_print_partial_info(f) */
  for (f = lista_filmes; f != NULL; f = (filme *)f->prox_filme) {
    da_print_partial_info(f);
    printf("\n-------------------------\n");
  }
	
  da_free_all(lista_filmes);
  printf("Tecle Enter para continuar..."); getchar();
  return;
}

/* ## 3 ## */
void client_reg_completo(int socketfd) {

  char c, id_procurado[TAM_REG_ID]; /* 20 */
  int i = 0;

  /* Leitura do id procurado (digito p/ dig.) */
  printf("ID do filme: ");
  c = getchar();
  while (c!='\n') { id_procurado[i] = c; i++; c = getchar(); }
  id_procurado[i] = '@'; /* coloca um @ para finalizar o id */

  /* envia o id procurado ao servidor */
  socket_push_buffer(socketfd, i+1, id_procurado);

  /* leitura da resposta do servidor */
  do {
    c = socket_pop_char(socketfd);
  } while(c == '\0'); /* limpa a stream */
  
  /* Caso não tenha encontrado nenhum filme */
  if (c == '#') {
    printf("\nFilme não encontrado.\n");
  } else {
    /* recebe a str do filme encontrado */
    char f_str[TAM_MAX_REG];
    i = 0;
    c = socket_pop_char(socketfd);
    while(c != '\0') { 
      f_str[i] = c;
      c = socket_pop_char(socketfd);
      i++;
    }

    /* monta a estrutura de filme */
    filme f;
    int tam_reg;
    da_str_to_filme(&f, &tam_reg, f_str);

    /* Imprime resultado da pesquisa */
    printf("Filme encontrado!\n\n");
    da_print_full_info(&f);
  }

  printf("\nAperte Enter para continuar...");
  getchar();

  return;
}

/* ## 4 ## */
void client_reg_sinopse(int socketfd) {

  /* Cópia do caso de uso para a listagem completa de um filme, 
     diferenciando apenas a impressão dos dados.*/
  
  char c, id_procurado[TAM_REG_ID]; int i = 0;

  printf("ID do filme: "); c = getchar();
  while (c!='\n') { id_procurado[i] = c; i++; c = getchar(); }
  id_procurado[i] = '@'; /* coloca um @ para finalizar o id */

  socket_push_buffer(socketfd, i+1, id_procurado);

  do { c = socket_pop_char(socketfd);
  } while(c == '\0'); /* limpa a stream */
  
  if (c == '#') { printf("\nFilme não encontrado.\n"); }
  else {
    char f_str[TAM_MAX_REG];
    i = 0; c = socket_pop_char(socketfd);
    while(c != '\0') { 
      f_str[i] = c; 
      c = socket_pop_char(socketfd);
      i++;
    }

    filme f; int tam_reg;
    da_str_to_filme(&f, &tam_reg, f_str);

    printf("Filme encontrado!\n\n");
    printf("Sinopse: %s\n", f.sinopse);
  }

  printf("\nAperte Enter para continuar...");
  getchar();

  return;
}

/* ## 5 ## */
void client_reg_media(int socketfd) {

  /* Cópia do caso de uso para a listagem completa de um filme, 
     diferenciando apenas a impressão dos dados.*/

  char c, id_procurado[TAM_REG_ID]; int i = 0;

  printf("ID do filme: "); c = getchar();
  while (c!='\n') { id_procurado[i] = c; i++; c = getchar(); }
  id_procurado[i] = '@'; /* coloca um @ para finalizar o id */

  socket_push_buffer(socketfd, i+1, id_procurado);

  do { c = socket_pop_char(socketfd);
  } while(c == '\0'); /* limpa a stream */
  
  if (c == '#') { printf("\nFilme não encontrado.\n"); }
  else {
    char f_str[TAM_MAX_REG];
    i = 0; c = socket_pop_char(socketfd);
    while(c != '\0') { 
      f_str[i] = c; 
      c = socket_pop_char(socketfd);
      i++;
    }

    filme f; int tam_reg;
    da_str_to_filme(&f, &tam_reg, f_str);

    printf("Filme encontrado!\n\n");
    printf("Média: %3.2f (%d avaliações)\n", f.media, f.n_aval);
  }

  printf("\nAperte Enter para continuar...");
  getchar();

  return;
}

/* ## 6 ## */
void client_reg_avalia(int socketfd) {

  char c, id_avaliar[TAM_REG_ID] /*20*/, nota[7];
  int i, j;

  /* Leitura do id do filme para avaliar e nota */
  printf("ID do filme a avaliar: ");
  i = 0; c = getchar();
  while (c!='\n') { id_avaliar[i] = c; i++; c = getchar(); }
  id_avaliar[i] = '@'; /* coloca um @ para finalizar o id */
  printf("Nota [formato: abc.de]: ");
  j = 0; c = getchar();
  while (c!='\n') { nota[j] = c; j++; c = getchar(); }
  nota[j] = '@'; /* coloca um @ para finalizar a nota*/

  /* envia o id e a nota ao servidor */
  socket_push_buffer(socketfd, i+1, id_avaliar);
  socket_push_buffer(socketfd, j+1, nota);

  /* leitura da resposta do servidor */
  do {
    c = socket_pop_char(socketfd);
  } while(c == '\0'); /* limpa a stream */
  
  /* Caso não tenha encontrado nenhum filme */
  if (c == '#') {
    printf("\nFilme não encontrado.\n");
  } else {
    printf("\nAvaliação realizada com sucesso!\n");
  }

  printf("\nAperte Enter para continuar...");
  getchar();

  return;
}

/*******[fim] Funções que implementam os casos de uso  ********/
/**************************************************************/



int main(int argc, char** argv) {

  /* Caso não haja o nome do servidor, da um erro */
  if (argc != 2) {
    fprintf(stderr, "uso: ./client <nome do servidor>\n");
    exit(1);
  }
	
  /* Estabelece a conexão com o servidor */
  int socketfd; //Socket de conexao
  socketfd = client_get_connection(argv);

  /* Loop da interface e chamadas para as funções que implementam cada 
     uso do sistema. */
  char c;
		
  c = read_option();
    
  /* Envia a opção escolhida ao servidor (mesmo se for Sair) */
  client_send_option(socketfd, c);

  while(c != SAIR) {
      
    switch(c) {
	
    case LISTAR_TODOS_COMPLETO:
      client_lista_todos_completo(socketfd);
      break;
    case LISTAR_TODOS:
      client_lista_todos(socketfd);
      break;
    case REG_COMPLETO:
      client_reg_completo(socketfd);
      break;
    case REG_SINOPSE:
      client_reg_sinopse(socketfd);
      break;
    case REG_MEDIA:
      client_reg_media(socketfd);
      break;
    case REG_AVALIAR:
      client_reg_avalia(socketfd);
      break;
			
    }
    
    c = read_option();
    client_send_option(socketfd, c);

  }
	
  close(socketfd); // fecha a conexão com o servidor
  return(0);

}
\end{verbatim}


%%%%%%%%%%%%%%%%%%%%%%%%%%%%%%%%%%%%%%%%%%
%%%%%%% Makefile
\subsection{Makefile}       % 9 
\begin{verbatim}
# Variáveis
CC = gcc
CC_FLAGS = -ggdb -Wall


# Dependências gerais
all: client server relatorio

# Cliente
client: data_access.o internet.o client.o
  $(CC) $(CC_FLAGS) data_access.o internet.o client.o -o client

client.o: client.c data_access.h internet.h defines.h
  $(CC) $(CC_FLAGS) -c client.c

# Servidor
server: data_access.o internet.o server.o
  $(CC) $(CC_FLAGS) -pthread data_access.o internet.o server.o -o server

server.o: server.c data_access.h internet.h defines.h
  $(CC) $(CC_FLAGS) -pthread -c server.c

# Bibliotecas
data_access.o: data_access.c data_access.h defines.h
  $(CC) $(CC_FLAGS) -c data_access.c

internet.o: internet.c internet.h defines.h
  $(CC) $(CC_FLAGS) -c internet.c


# Relatório (LaTeX)
relatorio:
  pdflatex relatorio.tex

# Clean
clean:
  rm *.o client server
  rm *.aux *.toc *.pdf *.log
\end{verbatim}



\begin{thebibliography}{99}
\bibitem{R1} HALL, Brian. Beej's Guide to Network Programming. Disponível em: http://http://beej.us/guide/bgnet/output/html/multipage/index.html.
\end{thebibliography}

\end{document}
